Marius Abstract

Das Internet of Things - IoT - beschäftigt sich mit der Vernetzung von Dingen. Diese Dinge, im Englischen Things genannt, beschreiben ein physikalisch, eindeutig identifizierbares Gerät. Das Ziel hierbei ist, die Mensch-Maschinen-Interaktion zu verringern.  Aufgrund der verschiedenen Zustände, die ein Mensch oder eine Maschine haben kann, gilt es dieses Hindernis zu überwinden. Dies(Bezugswort fehlt) soll durch verschiedene Teilziele erreicht werden. Zum einen spielen einheitliche Standard eine wichtige Rolle, sodass die sich im Netz befindenden Geräte die gleiche Grundlage für die Kommunikation haben. Zum anderen ist eine flächendeckende Netzabdeckung ebenfalls wichtig, da so viele Geräte wie möglich ihre Informationen austauschen können sollen. Zuletzt ist die Entwicklung von Internet of Things Services von großer Bedeutung - viele Informationen nützen nicht ohne dass aus denen die richtigen Erkenntnis gezogen und Interaktionen gestartet werden. Die Infrastruktur kann sich vielseitig zeigen: Zum einen gibt es dezentrale und zentral Ansätze, zum anderen gibt es verschiedene Leitungsmedien, in denen die Information übermittelt wird. Die Geräte, auch Teilnehmer genannt, können aktiv ihre Zustände mitteilen wie zum Beispiel bei WLAN. Es sind allerdings auch passive Alternativen vorhanden, durch den Einsatz von RFID.
